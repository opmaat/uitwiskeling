\documentclass{ximera}
\input{../../preamble}
\addPrintStyle{..}
\begin{document}
	\author{Wiskundeplan}
	\xmtitle{Oefeningen niveau 1}{}
	



\begin{exercise} Bereken de afgeleide 
    \begin{xmmulticols}

    \begin{question} \( \afg{  \sin x  } =  \answer[onlineshowanswerbutton]{  \cos x }\) \end{question}
    \begin{question} \( \afg{  \cos x  } =  \answer[onlineshowanswerbutton]{ -\sin x }\) \end{question}
    \begin{question} \( \afg{ -\sin x  } =  \answer[onlineshowanswerbutton]{ -\cos x }\) \end{question}
    \begin{question} \( \afg{ -\cos x  } =  \answer[onlineshowanswerbutton]{  \sin x }\) \end{question}
    \begin{question} \( \afg{  2\sin x  } =  \answer[onlineshowanswerbutton]{  2\cos x }\) \end{question}
    \begin{question} \( \afg{  6\cos x  } =  \answer[onlineshowanswerbutton]{ -6\sin x }\) \end{question}
    \begin{question} \( \afg{ - \pi \sin x  } =  \answer[onlineshowanswerbutton]{ - \pi \cos x }\) \end{question}
    \begin{question} \( \afg{ - 4\cos x  } =  \answer[onlineshowanswerbutton]{  4\sin x }\) \end{question}
    
    \end{xmmulticols}
\end{exercise}

\begin{exercise} Bereken de afgeleide 
    \begin{xmmulticols}
        \begin{question} \( \afg{ \sin x + \cos x       }= \answer[onlineshowanswerbutton]{\cos x - \sin x        } \) \end{question}
        \begin{question} \( \afg{ \sin x \cdot \cos x   }= \answer[onlineshowanswerbutton]{\cos^2 x - \sin^2 x    } \) \end{question}
        \begin{question} \( \afg{ \frac{\sin x}{\cos x} }= \answer[onlineshowanswerbutton]{\sec^2 x               } \) \end{question}
        \begin{question} \( \afg{ \frac{\cos x}{\sin x} }= \answer[onlineshowanswerbutton]{-\csc^2 x              } \) \end{question}
        \begin{question} \( \afg{ \sin^2 (x)            }= \answer[onlineshowanswerbutton]{2\sin x \cos x         } \) \end{question}
        \begin{question} \( \afg{ \sin (2x)             }= \answer[onlineshowanswerbutton]{2\cos(2x)              } \) \end{question}
        \begin{question} \( \afg{ \frac{1}{\sin x}      }= \answer[onlineshowanswerbutton]{-\csc x \cot x         } \) \end{question}
        \begin{question} \( \afg{ \frac{1}{\cos x }     }= \answer[onlineshowanswerbutton]{ \sec x \tan x         } \) \end{question}
        
    \end{xmmulticols}
\end{exercise}


\begin{exercise} Bereken de afgeleide
    \begin{xmmulticols}
        \begin{question} \( \afg{3 \sin^2 x               } = \answer[onlineshowanswerbutton]{6 \sin x \cos x                                         } \) \end{question}
        \begin{question} \( \afg{3 \cos^2 x               } = \answer[onlineshowanswerbutton]{ -6 \cos x \sin x                                       } \) \end{question}
        \begin{question} \( \afg{\cos(\sin x)             } = \answer[onlineshowanswerbutton]{ -\sin(\sin x) \cos x                                   } \) \end{question}
        \begin{question} \( \afg{\sin(\cos x)             } = \answer[onlineshowanswerbutton]{ -\cos(\cos x) \sin x                                   } \) \end{question}
        \begin{question} \( \afg{\cos(\cos x)             } = \answer[onlineshowanswerbutton]{ \sin(\cos x) \sin x                                    } \) \end{question}
        \begin{question} \( \afg{\tan(3x+1)               } = \answer[onlineshowanswerbutton]{ 3\sec^2(3x + 1)                                        } \) \end{question}
        \begin{question} \( \afg{\sin(2x) \cdot \sin x    } = \answer[onlineshowanswerbutton]{ \cos(2x) \cdot \sin x + \sin(2x) \cdot \cos x          } \) \end{question}
        \begin{question} \( \afg{\cos(x^2) \cdot \tan(x^2)} = \answer[onlineshowanswerbutton]{ -2x \sin(x^2) \tan(x^2) + 2x \cos(x^2) \sec^2(x^2)     } \) \end{question}
        
    \end{xmmulticols}
\end{exercise}



\end{document}



 
 
 

