\documentclass{ximera}
\input{../../preamble}
\addPrintStyle{..}
\begin{document}
	\author{Wiskundeplan}
	\xmtitle{Oefeningen niveau 2}{}
	



\begin{exercise} Bereken de afgeleide 
    \begin{xmmulticols}
    \begin{question} \( \tan^2 x                 =\answer[onlineshowanswerbutton]{ 2\tan x \cdot \frac{1}{\cos^2 x}    } \) \end{question}
    \begin{question} \( \sin^3 x                 =\answer[onlineshowanswerbutton]{ 3\sin^2 x \cos x                    } \) \end{question}
    \begin{question} \( \cos^3 x                 =\answer[onlineshowanswerbutton]{ -3\cos^2 x \sin x                   } \) \end{question}
    \begin{question} \( \frac{1}{\tan x}         =\answer[onlineshowanswerbutton]{ -\frac{1}{\sin^2 x}                 } \) \end{question}
    \begin{question} \( \frac{1}{\sin^2 x}       =\answer[onlineshowanswerbutton]{ -2\cos x \sin^{-3} x                } \) \end{question}
    \begin{question} \( \frac{1}{\cos^2 x}       =\answer[onlineshowanswerbutton]{ 2\sin x \cos^{-3} x                 } \) \end{question}
    \begin{question} \( \tan(2x)                 =\answer[onlineshowanswerbutton]{ 2\sec^2 (2x)                        } \) \end{question}
    \begin{question} \( \tan(x^2)                =\answer[onlineshowanswerbutton]{ \frac{2x}{\cos^2(x^2)}              } \) \end{question}
    \begin{question} \( \sin(3x) \cdot \cos(3x)  =\answer[onlineshowanswerbutton]{ 3\cos(6x)                           } \) \end{question}
    \begin{question} \( \cos(2x) \cdot \tan(2x)  =\answer[onlineshowanswerbutton]{ 2\sin(2x) + 2\cos^2(2x)/\cos^2(2x)  } \) \end{question}
    \end{xmmulticols}
\end{exercise}


\begin{exercise} Bereken de afgeleide 
    \begin{xmmulticols}
    \begin{question} \( \sin(4x+1)               =\answer[onlineshowanswerbutton]{ 4\cos(4x+1)                                        } \) \end{question}
    \begin{question} \( \cos(5x-3)               =\answer[onlineshowanswerbutton]{ -5\sin(5x-3)                                       } \) \end{question}
    \begin{question} \( \sin(\cos x)             =\answer[onlineshowanswerbutton]{ -\cos(\cos x) \sin x                               } \) \end{question}
    \begin{question} \( \cos(\sin x)             =\answer[onlineshowanswerbutton]{ -\sin(\sin x) \cos x                               } \) \end{question}
    \begin{question} \( \tan(\tan x)             =\answer[onlineshowanswerbutton]{ \frac{1}{\cos^2 x} \cdot \frac{1}{\cos^2 (\tan x)} } \) \end{question}
    \begin{question} \( \sin^2(\cos x)           =\answer[onlineshowanswerbutton]{ -2\cos(\cos x) \sin(\cos x) \sin x                 } \) \end{question}
    \begin{question} \( \cos^2(\sin x)           =\answer[onlineshowanswerbutton]{ -2\sin(\sin x) \cos(\sin x) \cos x                 } \) \end{question}
    \begin{question} \( x\sin x                  =\answer[onlineshowanswerbutton]{ \sin x + x\cos x                                   } \) \end{question}
    \begin{question} \( x\cos x                  =\answer[onlineshowanswerbutton]{ \cos x - x\sin x                                   } \) \end{question}
    \begin{question} \( x\tan x                  =\answer[onlineshowanswerbutton]{ \tan x + x\sec^2 x                                 } \) \end{question}
    
    \end{xmmulticols}
\end{exercise}


\begin{exercise} Bereken de afgeleide 
    \begin{xmmulticols}
        
    \begin{question} \( \sin x + x^2     =\answer[onlineshowanswerbutton]{ \cos x + 2x                   } \) \end{question}
    \begin{question} \( \cos x + x^3     =\answer[onlineshowanswerbutton]{ -\sin x + 3x^2                } \) \end{question}
    \begin{question} \( \sin^2 x + x^4   =\answer[onlineshowanswerbutton]{ 2\sin x\cos x + 4x^3          } \) \end{question}
    \begin{question} \( \cos^2 x + x^5   =\answer[onlineshowanswerbutton]{ -2\sin x\cos x + 5x^4         } \) \end{question}
    \begin{question} \( x\sin^2 x        =\answer[onlineshowanswerbutton]{ \sin 2x + 2x\sin x\cos x      } \) \end{question}
    \begin{question} \( x\cos^2 x        =\answer[onlineshowanswerbutton]{ -\sin 2x + 2x\cos x\sin x     } \) \end{question}
    \begin{question} \( \sin(2x) + x^2   =\answer[onlineshowanswerbutton]{ 2\cos(2x) + 2x                } \) \end{question}
    \begin{question} \( \cos(3x) + x^3   =\answer[onlineshowanswerbutton]{ -3\sin(3x) + 3x^2             } \) \end{question}
    \begin{question} \( \tan(x^2 + 1)    =\answer[onlineshowanswerbutton]{ \frac{2x}{\cos^2(x^2 + 1)}    } \) \end{question}
    \begin{question} \( \sin^2(3x)       =\answer[onlineshowanswerbutton]{ 6\sin(3x)\cos(3x)             } \) \end{question}
  
    \end{xmmulticols}
\end{exercise}


\begin{exercise} Bereken de afgeleide 
    \begin{xmmulticols}
    \begin{question} \( \cos^2(4x)               =\answer[onlineshowanswerbutton]{ -8\sin(4x)\cos(4x)                    } \) \end{question}
    \begin{question} \( \sin(x) \cdot \cos(x^2)  =\answer[onlineshowanswerbutton]{ \cos x \cos(x^2) - 2x\sin x\sin(x^2)  } \) \end{question}
    \begin{question} \( \sin(x^2) \cdot \cos x   =\answer[onlineshowanswerbutton]{ 2x\cos(x^2) \cos x - \sin(x^2) \sin x } \) \end{question}
    \begin{question} \( \sin(\cos(x^2))          =\answer[onlineshowanswerbutton]{ -2x\sin x \cos(\cos(x^2))             } \) \end{question}
    \begin{question} \( \cos(\sin(x^3))          =\answer[onlineshowanswerbutton]{ -3x^2\cos x \sin(\sin(x^3))           } \) \end{question}
    \begin{question} \( x^2\sin x                =\answer[onlineshowanswerbutton]{ 2x\sin x + x^2\cos x                  } \) \end{question}
    \begin{question} \( x^3\cos x                =\answer[onlineshowanswerbutton]{ 3x^2\cos x - x^3\sin x                } \) \end{question}
    \begin{question} \( x^4\sin(2x)              =\answer[onlineshowanswerbutton]{ 4x^3\sin(2x) + 2x^4\cos(2x)           } \) \end{question}
    \begin{question} \( x^5\cos(3x)              =\answer[onlineshowanswerbutton]{ 5x^4\cos(3x) - 3x^5\sin(3x)           } \) \end{question}
    \begin{question} \( \frac{x^2}{\sin x}       =\answer[onlineshowanswerbutton]{ \frac{2x\sin x - x^2\cos x}{\sin^2 x} } \) \end{question}  
    \end{xmmulticols}
\end{exercise}


\begin{exercise} Bereken de afgeleide 
    \begin{xmmulticols}       
    \begin{question} \( \frac{\tan x}{x}         = \answer[onlineshowanswerbutton]{ \frac{x\sec^2 x - \tan x}{x^2}                 } \) \end{question}
    \begin{question} \( \frac{x^3}{\cos x}       = \answer[onlineshowanswerbutton]{ \frac{3x^2\cos x + x^3\sin x}{\cos^2 x}        } \) \end{question}
    \begin{question} \( \frac{x}{\sin x}         = \answer[onlineshowanswerbutton]{ \frac{\sin x - x\cos x}{\sin^2 x}              } \) \end{question}
    \begin{question} \( \frac{x^2}{\cos x}       = \answer[onlineshowanswerbutton]{ \frac{2x\cos x + x^2\sin x}{\cos^2 x}          } \) \end{question}
    \begin{question} \( \frac{x^3}{\sin x}       = \answer[onlineshowanswerbutton]{ \frac{3x^2\sin x - x^3\cos x}{\sin^2 x}        } \) \end{question}
    \begin{question} \( \frac{x}{\cos x}         = \answer[onlineshowanswerbutton]{ \frac{\cos x + x\sin x}{\cos^2 x}              } \) \end{question}
    \begin{question} \( \frac{x^4}{\sin(2x)}     = \answer[onlineshowanswerbutton]{ \frac{4x^3\sin(2x) - 2x^4\cos(2x)}{\sin^2(2x)} } \) \end{question}
    \begin{question} \( \frac{x^5}{\cos(2x)}     = \answer[onlineshowanswerbutton]{ \frac{5x^4\cos(2x) + 2x^5\sin(2x)}{\cos^2(2x)} } \) \end{question}
    \begin{question} \( \tan(x^3 + x)            = \answer[onlineshowanswerbutton]{ \frac{(3x^2 + 1)}{\cos^2(x^3 + x)}             } \) \end{question}
    \begin{question} \( x^2 \sin(x^2 + 1)        = \answer[onlineshowanswerbutton]{ 2x\sin(x^2 + 1) + 2x^3\cos(x^2 + 1)            } \) \end{question}   
    \end{xmmulticols}
\end{exercise}




\end{document}