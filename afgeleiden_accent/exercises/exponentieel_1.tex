\documentclass{ximera}
\input{../../preamble}
\addPrintStyle{..}
\begin{document}
	\author{Wiskundeplan}
	\xmtitle{Oefeningen niveau 1}{}
	



\begin{exercise} Bereken de afgeleide
    \begin{xmmulticols}

    \begin{question} \( \afg{ e^x  }          =  \answer[onlineshowanswerbutton]{ e^x     }\) \end{question}
    \begin{question} \( \afg{ 2e^x  }         =  \answer[onlineshowanswerbutton]{ 2e^x    }\) \end{question}
    \begin{question} \( \afg{ 3e^x  }         =  \answer[onlineshowanswerbutton]{ 3e^x    }\) \end{question}
    \begin{question} \( \afg{ e^x +1  }       =  \answer[onlineshowanswerbutton]{ e^x     }\) \end{question}
    \begin{question} \( \afg{ e^x +2  }       =  \answer[onlineshowanswerbutton]{ e^x     }\) \end{question}
    \begin{question} \( \afg{ e^x + e^x  }    =  \answer[onlineshowanswerbutton]{ 2e^x    }\) \end{question}
    \begin{question} \( \afg{ 3e^x - 10e^x  } =  \answer[onlineshowanswerbutton]{ -7e^x   }\) \end{question}
    \begin{question} \( \afg{ e^{\pi} }       =  \answer[onlineshowanswerbutton]{ 0       }\) \end{question}
    \begin{question} \( \afg{ \pi e^x}        =  \answer[onlineshowanswerbutton]{ \pi e^x }\) \end{question}
    
    \end{xmmulticols}    
\end{exercise}



\begin{exercise} Bereken de afgeleide 
    \begin{xmmulticols}
    \begin{question} \( \afg{ e^2x        } =  \answer[onlineshowanswerbutton]{ 2e^{2x}      }\) \end{question}
    \begin{question} \( \afg{ e^{-x}      } =  \answer[onlineshowanswerbutton]{ -e^{-x}      }\) \end{question}
    \begin{question} \( \afg{ e^{x^2}     } =  \answer[onlineshowanswerbutton]{ 2xe^{x^2}    }\) \end{question}
    \begin{question} \( \afg{ xe^x        } =  \answer[onlineshowanswerbutton]{ (1+x)e^x     }\) \end{question}
    \begin{question} \( \afg{ x^2e^x      } =  \answer[onlineshowanswerbutton]{ xe^x(2+x)    }\) \end{question}
    \begin{question} \( \afg{ e^{2x+1}    } =  \answer[onlineshowanswerbutton]{ 2e^{2x+1}    }\) \end{question}
    \begin{question} \( \afg{ e^{\pi}e^x  } =  \answer[onlineshowanswerbutton]{ e^{\pi}e^x   }\) \end{question}
    \begin{question} \( \afg{ e^{\sin x } } =  \answer[onlineshowanswerbutton]{ \cos(x) e^x  }\) \end{question}
    \begin{question} \( \afg{ e^{\cos x } } =  \answer[onlineshowanswerbutton]{ -\sin(x) e^x }\) \end{question}    
    \end{xmmulticols}    
\end{exercise}


\begin{exercise}
    \begin{question} \( \afg{ \ln x         } =\answer[onlineshowanswerbutton]{ \frac{1}{x}           }\) \end{question} 
    \begin{question} \( \afg{ \ln(2x)       } =\answer[onlineshowanswerbutton]{ \frac{1}{x}           }\) \end{question}
    \begin{question} \( \afg{ \ln(x^2)      } =\answer[onlineshowanswerbutton]{ \frac{2}{x}           }\) \end{question}
    \begin{question} \( \afg{ \ln(3x + 1)   } =\answer[onlineshowanswerbutton]{ \frac{3}{3x + 1}      }\) \end{question}
    \begin{question} \( \afg{ \ln(x^3 + 4)  } =\answer[onlineshowanswerbutton]{ \frac{3x^2}{x^3 + 4}  }\) \end{question}
    \begin{question} \( \afg{ x \ln x       } =\answer[onlineshowanswerbutton]{ \ln x + 1             }\) \end{question}
    \begin{question} \( \afg{ x^2 \ln x     } =\answer[onlineshowanswerbutton]{ 2x \ln x + x          }\) \end{question}
    \begin{question} \( \afg{ \ln(\sin x)   } =\answer[onlineshowanswerbutton]{ \frac{\cos x}{\sin x} }\) \end{question}
    \begin{question} \( \afg{ \ln(x^2 + 1)  } =\answer[onlineshowanswerbutton]{ \frac{2x}{x^2 + 1}    }\) \end{question}
    \begin{question} \( \afg{ \ln(e^x)      } =\answer[onlineshowanswerbutton]{ 1                     }\) \end{question}    
\end{exercise}
    
\end{document}