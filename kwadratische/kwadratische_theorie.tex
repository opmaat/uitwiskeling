\documentclass{ximera}
\input{../preamble}
\addPrintStyle{..}
\begin{document}
	\author{Wiskunde Op Maat}
	\xmtitle{Kwadratische vergelijkingen oplossen}{}





\begin{proposition}Vierkantsvergelijkingen $ax^2+bx+c=0$ oplossen in $\R$.\nl

Een (reële) tweedegraadsvergelijking van de vorm $ax^2+bx+c$, met $a,b,c\in\R$ en $a\neq0$
heeft als                                                                                                                \textbf{discriminant} het (reële) getal  \important{D=b^2-4ac}, 
en heeft als oplossingen

    \begin{tabular}{lll}
        als $D<0$: & geen reële oplossingen \\
        als $D=0$: & precies een reële oplossing, namelijk \important{x_1=-\dfrac{b}{2a}} \\
        als $D>0$: & precies twee reële oplossingen, namelijk 
                        \important{x_1=\dfrac{-b+\sqrt{D}}{2a}} en \important{x_2=\dfrac{-b-\sqrt{D}}{2a}}
    \end{tabular}

Bovendien zijn de volgende uitspraken equivalent:

    \begin{tabular}{lll}
        (a) & $x_1$ en $x_2$ zijn oplossingen van de vergelijking $ax^2+bx+c=0$ \\
        (b) & $x_1$ en $x_2$ zijn nulpunten  van de functie $f(x)=ax^2+bx+c$ \\
        (c) & $x_1$ en $x_2$ zijn snijpunten van de kromme $y=ax^2+bx+c$ met de $x$-as \\
        (d) & $ax^2+bx+c = a(x-x_1)(x-x_2)$ & (ontbinden in factoren)\\
        (e) & $x_1+x_2 = -\dfrac{b}{a}$ en $x_1\cdot x_2 = \frac{c}{a}$ & (som en product van de wortels)
    \end{tabular}

\end{proposition}


\begin{example}
    \begin{question} \( f(x) = x^2 + 5x + 4 = \answer[onlineshowanswerbutton]{ -1                                        \text{ en } -4} \) 
        \begin{oplossing}           
            
            We beschouwen de kwadratische functie \(f(x) = x^2 + 5x + 4 \)
            
            De wortels van een kwadratische vergelijking van de vorm \( ax^2 + bx + c = 0 \) worden gevonden met de formule:
            
            \[x = \frac{-b \pm \sqrt{b^2 - 4ac}}{2a}\]
            
            waarbij in ons geval:
            
            \[a = 1, \quad b = 5, \quad c = 4\]
            
            Berekenen eerst de discriminant:
            
            \[ D = b^2 - 4ac = 5^2 - 4(1)(4) = 25 - 16 = 9 \]
            
            Aangezien de discriminant positief is, zijn er twee reële oplossingen:
            
            \[ x = \frac{-5 \pm \sqrt{9}}{2(1)} \]
            
            \[ x = \frac{-5 \pm 3}{2} \]
            
            Hieruit volgen de twee wortels:
            
            \[ x_1 = \frac{-5 + 3}{2} = \frac{-2}{2} = -1 \]
            
            \[ x_2 = \frac{-5 - 3}{2} = \frac{-8}{2} = -4 \]
            
            Dus de oplossingen van de vergelijking \( x^2 + 5x + 4 = 0 \) zijn:
            
            \[ x = -1 \quad                                                                                              \text{of} \quad x = -4 \]
            
            Ontbonden in factoren geeft dit: \(f(x) = x^2 + 5x + 4 = (x+1)(x+4) \)
            
            
        \end{oplossing}
            
    \end{question}

    \begin{question} \( f(x) = 3x^2 + 2x + \frac{1}{3} = \answer[onlineshowanswerbutton]{\frac{-1}{3}} \)
        
        \begin{oplossing}
        
        We beschouwen de kwadratische functie \( f(x) = 3x^2 + 2x + \frac{1}{3} \)

        De wortels van een kwadratische vergelijking van de vorm \( ax^2 + bx + c = 0 \) worden gevonden met de abc-formule:

        \[
        x = \frac{-b \pm \sqrt{b^2 - 4ac}}{2a}
        \]

        waarbij in ons geval:
        \[
        a = 3, \quad b = 2, \quad c = \frac{1}{3}
        \]

        We berekenen eerst de discriminant:

        \[
        D = b^2 - 4ac = 2^2 - 4(3) \left(\frac{1}{3}\right) = 4 - 4 = 0
        \]

        Aangezien de discriminant nul is, is er precies één oplossing:

        \[
        x = \frac{-2}{2(3)}
        \]

        \[
        x = \frac{-2}{6} = \frac{-1}{3}
        \]

        Dus de enige oplossing van de vergelijking \( 3x^2 + 2x + \frac{1}{3} = 0 \) is:

        \[
        x = -\frac{1}{3}
        \]

        Ontbonden in factoren geeft dit \( f(x) = 3x^2 + 2x + \frac{1}{3} = (x  + \frac{1}{3} )(x  + \frac{1}{3} ) = (x  + \frac{1}{3} )^2 \)

        \end{oplossing}
    \end{question}

    \begin{question} \( f(x) = -2x^2 + 3x - 5 = \answer[onlineshowanswerbutton]{                                         \text{Geen reële oplossingen}} \) 
        \begin{oplossing}
        
        We beschouwen de kwadratische functie \(f(x) = -2x^2 + 3x - 5 \) 
        

        De wortels van een kwadratische vergelijking van de vorm \( ax^2 + bx + c = 0 \) worden gevonden met de abc-formule:

        \[
        x = \frac{-b \pm \sqrt{b^2 - 4ac}}{2a}
        \]

        waarbij in ons geval:
        \[
        a = -2, \quad b = 3, \quad c = -5
        \]

        We berekenen eerst de discriminant:

        \[
        D = b^2 - 4ac = 3^2 - 4(-2)(-5) = 9 - 40 = -31
        \]

        Aangezien de discriminant negatief is, zijn er geen reële oplossingen.

        Dus de vergelijking \( -2x^2 + 3x - 5 = 0 \) heeft geen reële wortels.
        \end{oplossing}
    \end{question}
\end{example}


\begin{exercise}
  
    \begin{question} De wortels van \( \quad x^2 - 4x + 3    \) zijn  \( \quad \answer[onlinenoinput]{ 1            \text{ en                   }3  } \) \end{question}
    \begin{question} De wortels van \( \quad -2x^2 + 4x + 8  \) zijn  \( \quad \answer[onlinenoinput]{ -2           \text{ en                   } 4 } \) \end{question}
    \begin{question} De wortels van \( \quad 2x^2 - 8x + 16  \) zijn  \( \quad \answer[onlinenoinput]{  2           \text{ (dubbele wortel)     }   } \) \end{question}
    \begin{question} De wortels van \( \quad 2x^2 + 5x + 2   \) zijn  \( \quad \answer[onlinenoinput]{ -\frac{1}{2} \text{ en                   } -2} \) \end{question}
    \begin{question} De wortels van \( \quad 3x^2 - 6x + 3   \) zijn  \( \quad \answer[onlinenoinput]{ 1            \text{ (dubbele wortel)     }   } \) \end{question}
    \begin{question} De wortels van \( \quad x^2 + 4x + 5    \) zijn  \( \quad \answer[onlinenoinput]{              \text{ geen reële wortels   }   } \) \end{question}
    \begin{question} De wortels van \( \quad x^2 + 2x - 8    \) zijn  \( \quad \answer[onlinenoinput]{ -4           \text{ en                   } 2 } \) \end{question}
    \begin{question} De wortels van \( \quad 4x^2 + 12x + 9  \) zijn  \( \quad \answer[onlinenoinput]{ -\frac{3}{2} \text{(dubbele wortel    )  }   }  \) \end{question}
    \begin{question} De wortels van \( \quad 3x^2 + 6x + 3   \) zijn  \( \quad \answer[onlinenoinput]{  -1          \text{ (dubbele wortel)     }   } \) \end{question}
    \begin{question} De wortels van \( \quad -x^2 + 4x + 1   \) zijn  \( \quad \answer[onlinenoinput]{ -\frac{1}{2} \text{ en                   } 5 } \) \end{question}
    
\end{exercise}


\begin{exercise}
    
    \begin{question} De wortels van \( \quad x^2 + 2x + 2     \) zijn \( \quad \answer[onlinenoinput]{               \text{ geen reële wortels }             } \) \end{question}
    \begin{question} De wortels van \( \quad 2x^2 - 3x - 5    \) zijn \( \quad \answer[onlinenoinput]{ \frac{5}{2}   \text{ en                 } -1          } \) \end{question}
    \begin{question} De wortels van \( \quad 5x^2 + 4x - 1    \) zijn \( \quad \answer[onlinenoinput]{ \frac{1}{5}   \text{ en                 }     -1      } \) \end{question}
    \begin{question} De wortels van \( \quad -3x^2 + 12x - 12 \) zijn \( \quad \answer[onlinenoinput]{ 2             \text{(dubbele wortel     )}             } \) \end{question}
    \begin{question} De wortels van \( \quad x^2 + 6x + 5     \) zijn \( \quad \answer[onlinenoinput]{ -1            \text{ en                 } -5          } \) \end{question}
    \begin{question} De wortels van \( \quad 4x^2 + 4x + 10   \) zijn \( \quad \answer[onlinenoinput]{               \text{ geen reële wortels }             } \) \end{question}
    \begin{question} De wortels van \( \quad 4x^2 + 8x + 3    \) zijn \( \quad \answer[onlinenoinput]{ -\frac{3}{2}  \text{ en                 } -\frac{1}{2}} \) \end{question}
    \begin{question} De wortels van \( \quad x^2 - 2x + 1     \) zijn \( \quad \answer[onlinenoinput]{ 1             \text{(dubbele wortel     )}             } \) \end{question}
    \begin{question} De wortels van \( \quad 3x^2 - 2x - 8    \) zijn \( \quad \answer[onlinenoinput]{ 2             \text{ en                 } -\frac{4}{3}} \) \end{question}
    \begin{question} De wortels van \( \quad -x^2 + 5x - 6    \) zijn \( \quad \answer[onlinenoinput]{ 2             \text{ en                 } 3           } \) \end{question}

\end{exercise}


\begin{example} Substitutie 
    
\end{example}
    
\begin{exercise} Bepaal de oplossingen van volgende vergelijkingen door een geschikte substitutie toe te passen. 
    
\end{exercise}

\end{document}