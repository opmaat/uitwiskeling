\documentclass{ximera}
\input{../preamble}
\addPrintStyle{..}
\begin{document}
	\author{Wiskunde Op Maat}
	\xmtitle{Theorie lineaire vergelijkingen}{}



  \formulevb{\begin{array}{rl} x + a & = b \\ x + a - \red{a} & = b - \red{a} \\ x & = b - a \end{array}}{\begin{array}{rl} x + a & = b \\ x + a - \red{a} & = b - \red{a} \\ x & = b - a \end{array}}
      
%   \begin{array}{rl} x + 2 & = 5 \\ x + 2 - \red{2} & = 5 - \red{2} \\ x & = 5 - 2  \end{array}}
  
  
  \formulevb{\begin{array}{r@{ }l@{ }} x - a & = b \\ x - a + \red{a} & = b + \red{a} \\ x & = b + a \end{array}}{
      \begin{array}{r@{ }l@{ }} x - 4 & = 3 \\ x - 4 + \red{4} & = 3 + \red{4} \\ x & = 3 + 4  \end{array}}
      
      
      \end{document}
\end{proposition}

\begin{proposition} Vergelijkingen van de vorm \(ax = b\)

  \textbf{Een factor die van kant wisselt, draait om}

  \formulevb{\begin{array}{r@{ }l@{ }} ax  & = b \\ \frac{{ax}}{\red{a}} & = \frac{b}{\red{a}} \\ x & = \frac{b}{a} \end{array}}{
    \begin{array}{r@{ }l@{ }} 4x  & = 8 \\ \frac{4x}{\red{4}} & = \frac{8}{\red{4}} \\ x & = \frac{8}{4} \end{array}}

  \formulevb{\begin{array}{r@{ }l@{ }} \frac{x}{a}  & = b \\ \frac{\red{a}x}{a} & = \red{a}b \\ x & = ab \end{array}}{
    \begin{array}{r@{ }l@{ }} 4x  & = 8 \\ \frac{4x}{\red{4}} & = \frac{8}{\red{4}} \\ x & = \frac{8}{4} \end{array}}
 


\end{proposition}


\begin{exercise}

\begin{question} \( x+5         = 11  \answer[onlineshowanswerbutton]{ x = 6            } \) \end{question}
\begin{question} \( 3x          = 6   \answer[onlineshowanswerbutton]{ x = 2            } \) \end{question}
\begin{question} \( \frac{x}{2} = 7   \answer[onlineshowanswerbutton]{ x = 14           } \) \end{question}
\begin{question} \( x-1         = 0   \answer[onlineshowanswerbutton]{ x = 1            } \) \end{question}
\begin{question} \( 3x+2        = 0   \answer[onlineshowanswerbutton]{ x = \frac{-2}{3} } \) \end{question}
\begin{question} \( 7x+3        = 10  \answer[onlineshowanswerbutton]{ x = 1            } \) \end{question}
\begin{question} \( x+1         = 1   \answer[onlineshowanswerbutton]{ x = 0            } \) \end{question}
\begin{question} \( 4x+x        = -4  \answer[onlineshowanswerbutton]{ x = -2           } \) \end{question}
\begin{question} \( 3x-11       = 1   \answer[onlineshowanswerbutton]{ x = 4            } \) \end{question}
\begin{question} \( 2x+13       = 13  \answer[onlineshowanswerbutton]{ x = 0            } \) \end{question}

\end{exercise}


\begin{proposition} Vergelijkingen van de vorm \(ax + b  = cx + d\)

  \textbf{Alle termen die x bevatten naar 1 lid brengen, daarna x afzonderen}

  \formulevb{\begin{array}{r@{ }l@{ }} ax +b & = cx +d \\ ax - cx & = d-c \\ x(a-c) & = d-c \\ x & = \frac{d-c}{a-c} \end{array}}{
  \begin{array}{r@{ }l@{ }} -2x +1 & = x +4 \\ -2x - x & = 4-2 \\ x(-2-1) & = 4-2 \\ x & = \frac{4-2}{-2-1} \end{array}}

\end{proposition}


\begin{exercise}
\begin{question} \( 3x + 2  = 5         \answer[onlineshowanswerbutton]{ x = 1 } \) \end{question}
\begin{question} \( x + 7   = 2x - 5    \answer[onlineshowanswerbutton]{ x = 13} \) \end{question}
\begin{question} \( 2 x+2   = 4x + 4    \answer[onlineshowanswerbutton]{ x = -1} \) \end{question}
\begin{question} \( x       = -3 - 4    \answer[onlineshowanswerbutton]{ x = -1} \) \end{question}
\begin{question} \( 5x - 3  = 2x + 6    \answer[onlineshowanswerbutton]{ x = 3 } \) \end{question}
\begin{question} \( 4x + 7  = 3x - 2    \answer[onlineshowanswerbutton]{ x = -9} \) \end{question}
\begin{question} \( 6x - 4  = 2x + 8    \answer[onlineshowanswerbutton]{ x = 3 } \) \end{question}
\begin{question} \( 7x + 5  = 3x + 17   \answer[onlineshowanswerbutton]{ x = 3 } \) \end{question}
\begin{question} \( 8x - 6  = 2x + 12   \answer[onlineshowanswerbutton]{ x = 3 } \) \end{question}
\begin{question} \( 10x + 4 = 6x + 24   \answer[onlineshowanswerbutton]{ x = 5 } \) \end{question}
\begin{question} \( 9x - 3  = 2x + 11   \answer[onlineshowanswerbutton]{ x = 2 } \) \end{question}
\begin{question} \( 12x + 8 = 4x + 24   \answer[onlineshowanswerbutton]{ x = 2 } \) \end{question}
\begin{question} \( 5x + 3  = 2x + 12   \answer[onlineshowanswerbutton]{ x = 3 } \) \end{question}
\begin{question} \( 4x - 7  = 3x + 1    \answer[onlineshowanswerbutton]{ x = 8 } \) \end{question}
\begin{question} \( 11x + 9 = 5x + 21   \answer[onlineshowanswerbutton]{ x = 2 } \) \end{question}

\end{exercise}
\end{document}