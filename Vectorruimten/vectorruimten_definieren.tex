\documentclass{ximera}
\input{../preamble}
\addPrintStyle{..}
\begin{document}
\author{Alexander Holvoet}
\xmtitle{Vectorruimtes definiëren}{}

belangrijkste eigenschappen:
$$
r\cdot (\mathbf{u}+\mathbf{v})=r\cdot \mathbf{u}+r\cdot \mathbf{v}
\forall r,s\in \mathbb{R}:\forall \mathbf{u} \in V:(r+s)\cdot \mathbf{u}=r\cdot \mathbf{u}+s \cdot \mathbf{u}
\forall r,s\in \mathbb{R}:\forall \mathbf{u} \in V:r\cdot (s\cdot \mathbf{u})=(r\cdot s) \cdot \mathbf{u}
$$
IE coordinatie van vectoroptelling en scalaire vermenigvuldiging
kan je doen op zich
een vectorruimte is gewoon een verzameling waarbinnen je altijd blijft, geen uitbreiding nodig!

LAAT LEERLINGEN MET HEEL EXOTISCHE OPTELLING EN VERMENIGVULDIGING WERKEN

CONTROLEER DE EIGENSCHAPPEN VOOR CONCRETE VERZAMELING?

Je zou eventueel ook het definieren van die vectorruimten eerder kunnen plaatsen door eigenschappen van voortgebrachte delen te onderzoeken, maar dat is in zekere zin onnatuurlijker dan opmerken dat bij een lineaire transformatie \(f \mathbb{V} \to \mathbb{W}\) de deelruimten van \(V\) afgebeeld worden op deelruimten van \(W\).
Je kan daarvoor verderwerken uit de oplossingsverzameling van een homogeen lineair stelsel (wat altijd een vectorruimte is).
De onderzoekende leeractiviteit wordt dan gestuurd door ``Als deze twee vectoren een oplossing zijn van het homogeen stelsel, wat zijn dan alle andere oplossingen?''.


\end{document}