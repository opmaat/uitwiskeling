\documentclass{ximera}


\newcommand{\pijl}[1]{\vec{#1}}

\begin{document}
\author{Alexander Holvoet}
\xmtitle{Vectorruimtes definiëren}{}

\subsection*{Inleiding}
Met onderstaande oefeningen leren we kennen wat een `vectorruimte' is, aan de hand van `magische vierkanten'.
Een magisch vierkant is een vierkant rooster van \(n \times n\) gevuld met getallen, zodanig dat de som van de getallen in elke rij, elke kolom en beide diagonalen gelijk is.
De terugkerende som noemen we de `magische constante' \(C\).

\begin{exercise}
    Vul onderstaande tabel in zodat het een magisch vierkant wordt.
    \[
    \begin{array}{|c|c|c|}
    \hline
     &  &  \\
    \hline
    3 & 5 &  \\
    \hline
    4 &  & 2 \\
    \hline
    \end{array}
    \]
    \begin{hint}
        Je kan dit magisch vierkant invullen met de getallen \(1,2,\dots,9\).
    \end{hint}
    \begin{oplossing}
        De oplossing is:
        \[
        \begin{array}{|c|c|c|}
        \hline
        8 & 1 & 6 \\
        \hline
        3 & 5 & 7 \\
        \hline
        4 & 9 & 2 \\
        \hline
        \end{array}
        \]
        waarbij je ziet dat de magische constante \(C=15\).
    \end{oplossing}
\end{exercise}

Het voorgaande magische vierkant gebruikte enkel natuurlijke getallen.
Je kan echter ook magische vierkanten beschouwen met andere getallen, zoals in het volgende voorbeeld.

\begin{exercise}
    Vul onderstaande tabel in zodat het een magisch vierkant wordt.
    \[
    \begin{array}{|c|c|c|}
        \hline
        &  &  \\
        \hline
        2 &  &  \\
        \hline
        -1 &  & 0,5 \\
        \hline
    \end{array}
    \]
    \begin{hint}
        Duid goed aan op je figuur welke getallen je al gevonden hebt, of voor welke getallen je een nuttige vergelijking vondt.
        Het middenste getal kan je vinden met de vergelijking voor de eerste kolom en de diagonaal.
    \end{hint}
    \begin{hint}
        Controleer goed je vergelijkingen op welke variabelen gebruikt zijn, om te weten waar je kan substitueren.
        Nadat je het middenste getal vindt, kan je het middenste getal in de eerste rij vinden door de vergelijking voor kolom 2 en rij 3.
    \end{hint}
    \begin{hint}
        De magische constante voor dit vierkant is \(C=1,5\).
    \end{hint}
    \begin{oplossing}
        De oplossing is
        \[
        \begin{array}{|c|c|c|}
            \hline
            0.5 & -1 & 2 \\
            \hline
            2 & 0.5 & -1 \\
            \hline
            -1 & 2 & 0.5 \\
            \hline
        \end{array}
        \]
        Dit kon je terugvinden door zelf getallen te proberen en mogelijkheden uit te sluiten.
        Je kan ook vergelijkingen opstellen door de onbekenden een naam te geven:
        \[
        S=
        \begin{array}{|c|c|c|}
        \hline
        a & b & c \\
        \hline
        2 & d &  e\\
        \hline
        -1 & f & 0.5 \\
        \hline
        \end{array}
        \]
        De vereenvoudigde vergelijkingen zijn:
        \[
        \begin{cases}
            a + b + c &= C \\
            d + e &= C - 2 \\
            f &= C + 0.5 \\
            a &= C - 1 \\
            b + d + f &= C \\
            c + e &= C - 0.5 \\
            a + d &= C - 0.5 \\
            c + d &= C + 1
        \end{cases}
        \]
        Deze vergelijkingen kan je strategisch combineren om achtereenvolgens \(d=0,5\), \(b=-1\), \(c=2\) enzovoort te vinden.
    \end{oplossing}
\end{exercise}

Dergelijke magische vierkanten zijn al duizenden jaren oud, en men vroeg zich telkens af hoe je alle mogelijke magische vierkanten kan beschrijven.
Wij zullen in deze lesactiviteit spreken over magische vierkanten met decimale, negatieve, rationele, ... getallen. 
Kortom, de cellen in het vierkant mogen ingevuld worden met gelijk welk getal uit \(\mathbb{R}\).

\begin{exercise}
    Hoe zou je andere magische vierkanten van \(3 \times 3\) kunnen beschrijven?
    Schrijf je methode zorgvuldig uit.
    \begin{hint}
        Je kan variabelen \(s_{ij}\) gebruiken voor het getal in rij \(i\) en kolom \(j\) van het magische vierkant, en \(C\) voor de magische constante (de som van elke rij, kolom en diagonaal).
        \[
        S=
        \begin{array}{|c|c|c|}
        \hline
        s_{11} & s_{12} & s_{13} \\
        \hline
        s_{21} & s_{22} &  s_{23}\\
        \hline
        s_{31} & s_{32} & s_{33} \\
        \hline
        \end{array}
        \]
        Dat geeft je de volgende vergelijkingen:
        \[
        \begin{cases}
        s_{11} + s_{12} + s_{13} &= C \\
        s_{21} + s_{22} + s_{23} &= C \\
        s_{31} + s_{32} + s_{33} &= C \\
        s_{11} + s_{21} + s_{31} &= C \\
        s_{12} + s_{22} + s_{32} &= C \\
        s_{13} + s_{23} + s_{33} &= C \\
        s_{11} + s_{22} + s_{33} &= C \\
        s_{13} + s_{22} + s_{31} &= C
        \end{cases}
        \]  
    \end{hint}
    \begin{hint}
    Door de vergelijkingen voor de diagonaal, antidiagonaal en de tweede rij op te tellen, vinden we dat \(3s_{22}+s_{11}+s_{22}+s_{33}+s_{13}+s_{23}+s_{33}=3C\).
    Door de vergelijkingen voor de eerste en derde kolom toe te passen, vinden we uiteindelijk dat \(s_{22}=\frac{C}{3}\).
    \end{hint}
    \begin{hint}
        Door bijvoorbeeld de vergelijking voor de diagonaal te combineren met die voor de tweede kolom, krijgen we \(s_{11} + s_{33} = s_{12} + s_{32}\).
        Met de vergelijking voor de eerste rij en de derde kolom krijgen we \(s_{12} + s_{11} = s_{23} + s_{33}\).
        Deze twee vergelijkingen optellen geeft \(2s_{11} = s_{23} + s_{32}\).
        Je kan gelijkaardige vergelijkingen opstellen voor de cellen in de hoeken: \(s_{31}\), \(s_{13}\) en \(s_{33}\).
    \end{hint}
    \begin{oplossing}
        Je nam waarschijnlijk de algebraïsche benadering en stelde een stelsel vergelijkingen op.
        Het grote aantal variabelen en vergelijkingen maakt de situatie hier moeilijk.
        Alle vergelijkingen zijn gegeven in onderstaande stelsel:
        \[
        \begin{cases}
        s_{11} + s_{12} + s_{13} &= C \\
        s_{21} + s_{22} + s_{23} &= C \\
        s_{31} + s_{32} + s_{33} &= C \\
        s_{11} + s_{21} + s_{31} &= C \\
        s_{12} + s_{22} + s_{32} &= C \\
        s_{13} + s_{23} + s_{33} &= C \\
        s_{11} + s_{22} + s_{33} &= C \\
        s_{13} + s_{22} + s_{31} &= C
        \end{cases}
        \]
        Je kan eventueel overwegen om dit stelsel te rijreduceren met ICT, maar er bestaat een mooiere methode om veel magische vierkanten te beschrijven.      
    \end{oplossing}
\end{exercise}

In de vorige oefening probeerden we alle magische vierkanten te vinden door een groot stelsel vergelijkingen op te stellen.
Er bestaat echter een andere manier.
Als je al 1 magisch vierkant gevonden hebt, kan je misschien ook nog andere magische vierkanten daaruit terugvinden.
In de volgende oefening nemen we aan dat we al twee magische vierkanten gevonden hebben (zoals in de voorgaande oefeningen), en kijken we wat we daarmee nog kunnen doen.

\begin{exercise}
    Stel dat \(M\) en \(N\) twee magisch vierkanten van \(3\times 3\) zijn.
    Wat kan je dan zeggen over \(M + N\) en \(k \cdot M\)?
    Schrijf de aannames die je maakt duidelijk op, en beargumenteer je antwoord.
    \begin{oplossing}
        We nemen aan dat de optelling en scalaire vermenigvuldiging elementsgewijs gebeuren.
        Dan is zowel \(M + N\) als \(k \cdot M\) ook een magisch vierkant van \(3 \times 3\).

        Als de magische constante van \(M\) gelijk is aan \(C_M\) en die van \(N\) gelijk is aan \(C_N\), dan is de magische constante van \(M + N\) gelijk aan \(C_M + C_N\), en die van \(k \cdot M\) gelijk aan \(k \cdot C_M\).

        Merk op: deze redenering zou ook opgaan voor magische vierkanten van andere groottes, \(n \times n \) in het algemeen.
    \end{oplossing}
\end{exercise}

De magische vierkanten hebben gelijkaardige eigenschappen aan de vectoren in \(\mathbb{R}^2\); je kan ze namelijk ook optellen of vermenigvuldigen met een factor.
Dergelijke eigenschappen komen ook bij andere situaties voor, en dan spreken we over een `vectorruimte'.

\begin{definition}
    Een vectorruimte is een verzameling \(V\) met twee bewerkingen, optelling en scalaire vermenigvuldiging, die voldoen aan de volgende eigenschappen:
    \begin{enumerate}[label=(\arabic*)]
    \item $\forall \pijl{u},\pijl{v},\pijl{w} \in V:(\pijl{u}+\pijl{v})+\pijl{w}=\pijl{u}+(\pijl{v}+\pijl{w})$
    \item $\exists \pijl{0}_V \in V:\forall \pijl{u}\in V:\pijl{u}+\pijl{0}_V=\pijl{u}=\pijl{0}_V+\pijl{u}$
    \item $\forall \pijl{u}\in V:\exists \pijl{u'}\in V:\pijl{u}+\pijl{u'}=\pijl{0}_V=\pijl{u'}+\pijl{u}$
    \item $\forall \pijl{u},\pijl{v} \in V:\pijl{u}+\pijl{v}=\pijl{v}+\pijl{u}$
    \item $\forall r\in \mathbb{R}:\forall \pijl{u},\pijl{v} \in V:r\cdot (\pijl{u}+\pijl{v})=r\cdot \pijl{u}+r\cdot \pijl{v}$
    \item $\forall r,s\in \mathbb{R}:\forall \pijl{u} \in V:(r+s)\cdot \pijl{u}=r\cdot \pijl{u}+s \cdot \pijl{u}$
    \item $\forall r,s\in \mathbb{R}:\forall \pijl{u} \in V:r\cdot (s\cdot \pijl{u})=(r\cdot s) \cdot \pijl{u}$
    \item $\forall \pijl{u}\in V:1\cdot \pijl{u}=\pijl{u}$
    \end{enumerate}
\end{definition}

\begin{exercise}
    Vormt de verzameling van alle magische vierkanten van orde \(n\times n\) (met de elementsgewijze optelling en scalaire vermenigvuldiging) een vectorruimte?
    Controleer hiervoor stap voor stap iedere eigenschap, en schrijf op wat hij betekent.
    \begin{oplossing}
        \begin{enumerate}[label=(\arabic*)]
        \item $\forall \pijl{u},\pijl{v},\pijl{w} \in V:(\pijl{u}+\pijl{v})+\pijl{w}=\pijl{u}+(\pijl{v}+\pijl{w})$ (associativiteit van optelling)
        \item $\exists \pijl{0}_V \in V:\forall \pijl{u}\in V:\pijl{u}+\pijl{0}_V=\pijl{u}=\pijl{0}_V+\pijl{u}$ (neutraal element van optelling)
        \item $\forall \pijl{u}\in V:\exists \pijl{u'}\in V:\pijl{u}+\pijl{u'}=\pijl{0}_V=\pijl{u'}+\pijl{u}$ (bestaan van invers element voor optelling)
        \item $\forall \pijl{u},\pijl{v} \in V:\pijl{u}+\pijl{v}=\pijl{v}+\pijl{u}$ (commutativiteit optelling)
        \item $\forall r\in \mathbb{R}:\forall \pijl{u},\pijl{v} \in V:r\cdot (\pijl{u}+\pijl{v})=r\cdot \pijl{u}+r\cdot \pijl{v}$ (distributiviteit scalaire vermenigvuldiging en optelling)
        \item $\forall r,s\in \mathbb{R}:\forall \pijl{u} \in V:(r+s)\cdot \pijl{u}=r\cdot \pijl{u}+s \cdot \pijl{u}$ (distributiviteit scalaire vermenigvuldiging en optelling)
        \item $\forall r,s\in \mathbb{R}:\forall \pijl{u} \in V:r\cdot (s\cdot \pijl{u})=(r\cdot s) \cdot \pijl{u}$ (associativiteit scalaire vermenigvuldiging)
        \item $\forall \pijl{u}\in V:1\cdot \pijl{u}=\pijl{u}$ (neutraal element scalaire vermenigvuldiging)
    \end{enumerate}
    De voornaamste eigenschappen werden al in de vorige oefening gecontroleerd.
    Aangezien de optelling en scalaire vermenigvuldiging elementsgewijs gebeurt, zijn alle eigenschappen van de `gewone' optelling en vermenigvuldiging in \(\mathbb{R}\) van toepassing.
    Het neutraal element voor de optelling \(\pijl{0}_V\) is het vierkant met allemaal nullen; het inverse element wisselt het teken van ieder cel in het vierkant.   
    \end{oplossing}
\end{exercise}

Laat ons nu \(\mathbb{M}^{3\times 3}\) noteren voor de vectorruimte van magische vierkanten van orde \(3\times 3\).
Je hebt eerder gezien dat in de vectorruimte \(\mathbb{R}^2\) je iedere vector als een lineaire combinatie kon schrijven van goed gekozen vectoren.
Met bijvoorbeeld de vectoren \(\begin{pmatrix} 1 \\ 1 \end{pmatrix}\) en \(\begin{pmatrix} -1 \\ 3 \end{pmatrix}\) kan je iedere vector in \(\mathbb{R}^2\) schrijven:
\[
\begin{pmatrix} 2 \\ 6 \end{pmatrix} = 3\cdot \begin{pmatrix} 1 \\ 1 \end{pmatrix} + 1\cdot \begin{pmatrix} -1 \\ 3 \end{pmatrix}
\]
Je kan ook iets gelijkaardigs doen in de vectorruimte \(\mathbb{M}^{3\times 3}\)!

\begin{exercise}
    Zijn de onderstaande vierkanten ook magisch?
    \[
    A=
    \begin{array}{|c|c|c|}
    \hline
    1 & 1 & 1 \\
    \hline
    1 & 1 & 1 \\
    \hline
    1 & 1 & 1 \\
    \hline
    \end{array}
    , \quad
    B=
    \begin{array}{|c|c|c|}
    \hline
    0 & 1 & -1 \\
    \hline
    -1 & 0 & 1 \\
    \hline
    1 & -1 & 0 \\
    \hline
    \end{array}
    , \quad \text{en} \quad
    C=
    \begin{array}{|c|c|c|}
    \hline
    -1 & 1 & 0 \\
    \hline
    1 & 0 & -1 \\
    \hline
    0 & -1 & 1 \\
    \hline
    \end{array}
    \]
    Zo ja, wat zijn dan hun magische constanten?
    \begin{oplossing}
        Ja, alle drie de vierkanten zijn magisch.
        De constante voor \(A\) is \(3\), voor \(B\) is het \(0\) en voor \(C\) is het ook \(0\).
    \end{oplossing}
\end{exercise}

\begin{exercise}
    Zijn de onderstaande vierkanten lineair onafhankelijk?
    \[
    A=
    \begin{array}{|c|c|c|}
    \hline
    1 & 1 & 1 \\
    \hline
    1 & 1 & 1 \\
    \hline
    1 & 1 & 1 \\
    \hline
    \end{array}
    , \quad
    B=
    \begin{array}{|c|c|c|}
    \hline
    0 & 1 & -1 \\
    \hline
    -1 & 0 & 1 \\
    \hline
    1 & -1 & 0 \\
    \hline
    \end{array}
    , \quad \text{en} \quad
    C=
    \begin{array}{|c|c|c|}
    \hline
    -1 & 1 & 0 \\
    \hline
    1 & 0 & -1 \\
    \hline
    0 & -1 & 1 \\
    \hline
    \end{array}
    \]
    \begin{oplossing}
        Vierkant \(B\) en \(C\) hebben magische constante 0, dus ook hun lineaire combinaties zullen een magische constante van 0 hebben.
        Vierkant \(A\) heeft een magische constant van 3, en is dus zeker lineair onafhankelijk van \(B\) en \(C\).

        Dan rest nog de vraag of \(B\) en \(C\) lineair onafhankelijk zijn.
        Dat is ook het geval, want het magische vierkant \(k \cdot B\) zal altijd nullen hebben op de hoofddiagonaal, wat bij \(C\) niet het geval is.
        Je kan \(C\) dus niet maken uit \(B\) door een gepast veelvoud \(k\) te nemen.
        
        De drie matrices zijn met andere woorden lineair onafhankelijk.
    \end{oplossing}
\end{exercise}

\begin{exercise}
    Gebruikmakend van \(A\), \(B\) en \(C\) uit de vorige oefening, hoe kan je andere magische vierkanten vinden?
    En zijn hiermee alle magische vierkanten van orde \(3\times 3\) beschreven?
    \begin{oplossing}
        Je kan lineaire combinaties van \(A\), \(B\) en \(C\) nemen om andere magische vierkanten te vinden.
        Bijvoorbeeld, het magische vierkant uit de eerste oefening kan geschreven worden als \(5A - B - 3C\).
        In het algemeen, als je \(x\), \(y\) en \(z\) kiest, dan is \(xA + yB + zC\) een magisch vierkant met magische constante \(3x\).

        Het is ook omgekeerd het geval dat ieder magisch vierkant in \(\mathbb{M}^{3\times 3}\) een lineaire combinatie is van \(A\), \(B\), \(C\). 
        Dat kan je controleren door het voorgaande grote stelsel van vergelijkingen \href{https://math.stackexchange.com/questions/2137182/showing-a-set-of-matrices-is-a-basis-of-all-3x3-magic-squares}{uit te werken met ICT}.
    \end{oplossing}
\end{exercise}

\begin{exercise}
    Als we alleen de magische vierkanten van orde \(3\times 3\) met magische constante \(0\) zouden willen beschrijven, hoe kunnen we dat dan bereiken?
    \begin{oplossing}
        De magische vierkanten van orde \(3\times 3\) met magische constante \(0\) zijn ook een vectorruimte op zichzelf.
        Bij twee magische vierkanten \(M\), \(N\) met constante 0, is namelijk zowel \(M + N\) als \(k \cdot M\) ook een magisch vierkant met constante 0.
        Je kan ook de andere voorwaarden van een vectorruimte controleren.

        We kunnen dus een gelijkaardige redenering toepassen, zoals met \(A\), \(B\) en \(C\).
        Deze keer laten we \(A\) weg, want die heeft een magische constante van \(3\).
        Ieder magisch vierkant met magische constante 0 zal dus te schrijven zijn als \(yB + zC\).
    \end{oplossing}
\end{exercise}

\begin{exercise}
    Wat is het verband tussen
    \begin{itemize}
        \item de vectorruimte van magische vierkanten met magische constante \(0\): \(\mathbb{M}^{n\times n}_0\),
        \item de vectorruimte van alle magische vierkanten van \(3 \times 3\): \(\mathbb{M}^{3 \times 3}\) en
        \item de vectorruimte van alle \(3 \times 3\) matrices: \(\mathbb{R}^{3 \times 3}\) en
        \item de vectorruimte van alle \(n \times n\) matrices: \(\mathbb{R}^{n \times n}\)?
    \end{itemize}
    (allen voorzien van de gebruikelijke optelling en scalaire vermenigvuldiging)
    \begin{oplossing}
        We hebben de volgende inclusies:
        \begin{align*}
            \mathbb{M}^{n\times n}_0 &\subseteq \mathbb{R}^{n \times n}\\
            \mathbb{M}^{3 \times 3} &\subseteq \mathbb{R}^{3 \times 3}
        \end{align*}
        Andere verbanden zoals \(\mathbb{M}^{n\times n}_0 \subseteq \mathbb{M}^{3 \times 3}\) zijn er op zich niet onmiddellijk, aangezien de groottes van de magische vierkanten kunnen verschillen.
        Het vierkant \(
        \begin{array}{|c|c|}
        \hline
        0 & 0 \\
        \hline
        0 & 0 \\
        \hline
        \end{array}
        \)
        behoort bijvoorbeeld tot \(\mathbb{M}^{2\times 2}_0\), maar niet tot \(\mathbb{M}^{3 \times 3}\).
        Op dezelfde manier is bijvoorbeeld \(\mathbb{R}^2\) geen deelruimte van \(\mathbb{R}^{3}\).
    \end{oplossing}
\end{exercise}

\end{document}