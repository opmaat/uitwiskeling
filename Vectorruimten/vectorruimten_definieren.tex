\documentclass{ximera}
\input{../preamble}
\addPrintStyle{..}
\begin{document}
\author{Alexander Holvoet}
\xmtitle{Vectorruimtes definiëren}{}




% optional: https://en.wikipedia.org/wiki/Siamese_method cool method to include for investigation



OPWARM OEFENING:
Vul in onderstaande tabel zodat het een ``magisch vierkant'' wordt (de som van elke rij, kolom en diagonaal is gelijk).
\(
\begin{array}{|c|c|c|}
\hline
 &  &  \\
\hline
3 & 5 &  \\
\hline
4 &  & 2 \\
\hline
\end{array}
\)

Historical addition AS SOLUTION: + QUESTION: translate into system of equations?
\url{https://en.wikipedia.org/wiki/Magic_square#A_method_for_constructing_a_magic_square_of_order_3}

In the 19th century, Édouard Lucas devised the general formula for order 3 magic squares. Consider the following table made up of positive integers a, b and c:
\(
\begin{array}{|c|c|c|}
\hline
c - b & c + (a + b) & c - a \\
\hline
c - (a - b) & c & c + (a - b) \\
\hline
c + a & c - (a + b) & c + b \\
\hline
\end{array}
\)

These nine numbers will be distinct positive integers forming a magic square with the magic constant 3c so long as 0 < a < b < c - a and b $\neq$ 2a. Moreover, every 3$\times$3 magic square of distinct positive integers is of this form. 
We zullen het doen met order 3
Let A, B, C be the magic squares as in Example 1.1. Prove that all 3 $\times$ 3
magic squares can be written as $\lambda$A + $\mu$B + $\nu$C for some real numbers $\lambda$, $\mu$, $\nu$
\(
\begin{array}{|c|c|c|}
\hline
2 & 7 & 6 \\
\hline
9 & 5 & 1 \\
\hline
4 & 3 & 8 \\
\hline
\end{array}
\)

Check this, and verify that 5A - B - 3C equals the well-known magic square above.

BASIS FOR ONLY THE SUBSPACE OF BORING SUBSPACE WHERE IT ALWAYS SUMS TO
0
\(
\begin{array}{|c|c|c|}
\hline
0 & 1 & -1 \\
\hline
-1 & 0 & 1 \\
\hline
1 & -1 & 0 \\
\hline
\end{array}
\)
\(
\begin{array}{|c|c|c|}
\hline
-1 & 1 & 0 \\
\hline
1 & 0 & -1 \\
\hline
0 & -1 & 1 \\
\hline
\end{array}
\)
\(
\begin{array}{|c|c|c|}
\hline
1 & -1 & 0 \\
\hline
0 & 1 & -1 \\
\hline
-1 & 0 & 1 \\
\hline
\end{array}
\)

DAN ZELF: wat is het verband met veeltermen bijvoorbeeld?

LAAT LEERLINGEN MET HEEL EXOTISCHE OPTELLING EN VERMENIGVULDIGING WERKEN
ELS beste voorbeeldjes:
Set: V = $\mathbb{R}^{+}$ (positive real numbers)
Addition: x $\oplus$ y = xy (ordinary multiplication!)
Scalar multiplication: c $\odot$ x = $x^c$

OF matrices waarbij getallen optellen tot 1 getal

DEELRUIMTEN TRANSFORMATIES is ook goed idee om hier te vragen
Kzou nu eerst efkes het overkoepelende verhaal terug bekijken


CONTROLEER DE EIGENSCHAPPEN VOOR CONCRETE VERZAMELING?

belangrijkste eigenschappen:
$$
r\cdot (\mathbf{u}+\mathbf{v})=r\cdot \mathbf{u}+r\cdot \mathbf{v}
\forall r,s\in \mathbb{R}:\forall \mathbf{u} \in V:(r+s)\cdot \mathbf{u}=r\cdot \mathbf{u}+s \cdot \mathbf{u}
\forall r,s\in \mathbb{R}:\forall \mathbf{u} \in V:r\cdot (s\cdot \mathbf{u})=(r\cdot s) \cdot \mathbf{u}
$$
IE coordinatie van vectoroptelling en scalaire vermenigvuldiging
kan je doen op zich
een vectorruimte is gewoon een verzameling waarbinnen je altijd blijft, geen uitbreiding nodig!

Je zou eventueel ook het definieren van die vectorruimten eerder kunnen plaatsen door eigenschappen van voortgebrachte delen te onderzoeken, maar dat is in zekere zin onnatuurlijker dan opmerken dat bij een lineaire transformatie \(f \mathbb{V} \to \mathbb{W}\) de deelruimten van \(V\) afgebeeld worden op deelruimten van \(W\).
Je kan daarvoor verderwerken uit de oplossingsverzameling van een homogeen lineair stelsel (wat altijd een vectorruimte is).
De onderzoekende leeractiviteit wordt dan gestuurd door ``Als deze twee vectoren een oplossing zijn van het homogeen stelsel, wat zijn dan alle andere oplossingen?''.

\end{document}