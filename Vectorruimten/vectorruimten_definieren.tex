\documentclass{ximera}


\newcommand{\pijl}[1]{\vec{#1}}

\begin{document}
\author{Alexander Holvoet}
\xmtitle{Vectorruimtes definiëren}{}

\subsection*{Inleiding}
%%% KORT ONTWERP ACTIVITEIT Ximera, dan overnemen in tekst zelf
% - concreet magisch vierkant invullen
% - ander concreet magisch vierkant met rationele getallen vervolledigen
% - eigenschappen van vectorruimten controleren op R^2 zoals daarnet en magische vierkanten, zij zijn dus ook een "vector" op een abstracte manier
% - algebraisch proberen beschrijven van alle vierkanten orde 3, geef ook de rijgereduceerde matrix + doelbewust falen
% - vergelijking met R^n: daar konden we iedere vector schrijven als combinatie van twee specifieke vectoren (noem het geen basis?); kunnen we dat hier ook doen om alle magische vierkanten te beschrijven?
% - geef basis, laat ze onafhankelijkheid controleren zonder het zo te benoemen
% - laat ze concreet Ander magisch vierkant uitdrukken met die basis, vraag of het altijd terug een magisch vierkant is.
% - voortbrengend: geef intuitief argument waarom het alle vierkanten bereikt: je kan alle magische constantes hebben, en enige operaties om van gegeven vierkant een magischen te houden bij 1 constante, is om diagonaal te herbalanceren zoals in matrix A en B.
% - deelruimten: (is eerder sidetrack) laat ze deelruimte van M(3,3) vinden (constante 0) en overkoepelende ruimte Matrices 3x3.

Met onderstaande oefeningen leren we kennen wat een `vectorruimte' is, door zelfstandig met de eigenschappen geconfronteerd te worden.
Laat ons beginnen met een plezierig voorbeeld.\newline

Een magisch vierkant is een vierkant rooster van \(n \times n\) gevuld met getallen, zodanig dat de som van de getallen in elke rij, elke kolom en beide diagonalen gelijk is.
Soms vereist men ook dat alle getallen verschillend, positief en geheel zijn, maar dat zullen we hier niet doen.

\begin{exercise}
    Vul in onderstaande tabel zodat het een magisch vierkant wordt.
    \[
    \begin{array}{|c|c|c|}
    \hline
     &  &  \\
    \hline
    3 & 5 &  \\
    \hline
    4 &  & 2 \\
    \hline
    \end{array}
    \]
    \begin{oplossing}
        Eén mogelijke oplossing is:
        \[
        \begin{array}{|c|c|c|}
        \hline
        8 & 1 & 6 \\
        \hline
        3 & 5 & 7 \\
        \hline
        4 & 9 & 2 \\
        \hline
        \end{array}
        \]
    \end{oplossing}
\end{exercise}

\begin{exercise}
    Bestaat er een methode om andere magische vierkanten van \(3 \times 3\) te vinden?
    Schrijf je methode zorgvuldig uit.
    Indien gewenst, mag je gebruik maken van ICT, maar werk \textbf{maximaal} 10 minuten aan deze opdracht.
    \begin{hint}
        Je kan variabelen \(s_{ij}\) gebruiken voor het getal in rij \(i\) en kolom \(j\) van het magische vierkant, en \(C\) voor de magische constante (de som van elke rij, kolom en diagonaal).
        \[
        S=
        \begin{array}{|c|c|c|}
        \hline
        s_{11} & s_{12} & s_{13} \\
        \hline
        s_{21} & s_{22} &  s_{23}\\
        \hline
        s_{31} & s_{32} & s_{33} \\
        \hline
        \end{array}
        \]
        Dat geeft je de volgende vergelijkingen:
        \[
        \begin{cases}
        s_{11} + s_{12} + s_{13} &= C \\
        s_{21} + s_{22} + s_{23} &= C \\
        s_{31} + s_{32} + s_{33} &= C \\
        s_{11} + s_{21} + s_{31} &= C \\
        s_{12} + s_{22} + s_{32} &= C \\
        s_{13} + s_{23} + s_{33} &= C \\
        s_{11} + s_{22} + s_{33} &= C \\
        s_{13} + s_{22} + s_{31} &= C
        \end{cases}
        \]
    \end{hint}
    \begin{hint}
    Door de vergelijkingen voor de diagonaal, antidiagonaal en de tweede rij op te tellen, vinden we dat \(3s_{22}+s_{11}+s_{22}+s_{33}+s_{13}+s_{23}+s_{33}=3C\).
    Door de vergelijkingen voor de eerste en derde kolom toe te passen, vinden we uiteindelijk dat \(s_{22}=\frac{C}{3}\).
    \end{hint}
    \begin{hint}
        Door bijvoorbeeld de vergelijking voor de diagonaal te combineren met die voor de tweede kolom, krijgen we \(s_{11} + s_{33} = s_{12} + s_{32}\).
        Met de vergelijking voor de eerste rij en de derde kolom krijgen we \(s_{12} + s_{11} = s_{23} + s_{33}\).
        Deze twee vergelijkingen optellen geeft \(2s_{11} = s_{23} + s_{32}\).
        Je kan gelijkaardige vergelijkingen opstellen voor de cellen in de hoeken: \(s_{31}\), \(s_{13}\) en \(s_{33}\).
    \end{hint}
    \begin{oplossing}
        Je nam waarschijnlijk de algebraïsche benadering en stelde een stelsel vergelijkingen op.
        Het grote aantal variabelen en vergelijkingen maakt de situatie hier erg moeilijk.
        Je kan eventueel nog proberen enkele variabelen te elimineren zoals in de hints, maar het lijkt algauw hopeloos.\newline

        We zien in de volgende oefeningen een betere methode om magische vierkanten van \(3 \times 3\) te vinden.        
    \end{oplossing}
\end{exercise}

\begin{exercise}
    Stel dat \(M\) en \(N\) twee magisch vierkanten van \(3\times 3\) zijn.
    Wat kan je dan zeggen over \(M + N\) en \(k \cdot M\)?
    Schrijf de aannames die je maakt duidelijk op, en beargumenteer je antwoord.
    \begin{oplossing}
        We nemen aan dat de optelling en scalaire vermenigvuldiging elementsgewijs gebeurt.
        Dan is zowel \(M + N\) als \(k \cdot M\) ook een magisch vierkant van \(3 \times 3\).

        Als de magische constante van \(M\) gelijk is aan \(C_M\) en die van \(N\) gelijk is aan \(C_N\), dan is de magische constante van \(M + N\) gelijk aan \(C_M + C_N\), en die van \(k \cdot M\) gelijk aan \(k \cdot C_M\).

        Merk op: deze redenering zou ook opgaan voor magische vierkanten van andere groottes, \(n \times n \) in het algemeen.
    \end{oplossing}
\end{exercise}

\begin{exercise}
    Zijn de onderstaande vierkanten ook magisch?
    \[
    A=
    \begin{array}{|c|c|c|}
    \hline
    1 & 1 & 1 \\
    \hline
    1 & 1 & 1 \\
    \hline
    1 & 1 & 1 \\
    \hline
    \end{array}
    , \quad
    B=
    \begin{array}{|c|c|c|}
    \hline
    0 & 1 & -1 \\
    \hline
    -1 & 0 & 1 \\
    \hline
    1 & -1 & 0 \\
    \hline
    \end{array}
    , \quad \text{en} \quad
    C=
    \begin{array}{|c|c|c|}
    \hline
    -1 & 1 & 0 \\
    \hline
    1 & 0 & -1 \\
    \hline
    0 & -1 & 1 \\
    \hline
    \end{array}
    \]
    Zo ja, wat is dan de magische constante?
    \begin{oplossing}
        Ja, alle drie de vierkanten zijn magisch.
        De constante voor \(A\) is \(3\), voor \(B\) is het \(0\) en voor \(C\) is het ook \(0\).
    \end{oplossing}
\end{exercise}

\begin{exercise}
    Gebruikmakend van \(A\), \(B\) en \(C\) uit de vorige oefening, hoe kan je andere magische vierkanten vinden?
    \begin{oplossing}
        Je kan lineaire combinaties van \(A\), \(B\) en \(C\) nemen om andere magische vierkanten te vinden.
        Bijvoorbeeld, het magische vierkant uit de eerste oefening kan geschreven worden als \(5A - B - 3C\).
        In het algemeen, als je \(x\), \(y\) en \(z\) kiest, dan is \(xA + yB + zC\) een magisch vierkant met magische constante \(3x\).
    \end{oplossing}
\end{exercise}

\begin{definition}
    Een vectorruimte is een verzameling \(V\) met twee bewerkingen, optelling en scalaire vermenigvuldiging, die voldoen aan de volgende eigenschappen:
    \begin{enumerate}[label=(\arabic*)]
    \item $\forall \pijl{u},\pijl{v},\pijl{w} \in V:(\pijl{u}+\pijl{v})+\pijl{w}=\pijl{u}+(\pijl{v}+\pijl{w})$
    \item $\exists \pijl{0}_V \in V:\forall \pijl{u}\in V:\pijl{u}+\pijl{0}_V=\pijl{u}=\pijl{0}_V+\pijl{u}$
    \item $\forall \pijl{u}\in V:\exists \pijl{u'}\in V:\pijl{u}+\pijl{u'}=\pijl{0}_V=\pijl{u'}+\pijl{u}$
    \item $\forall \pijl{u},\pijl{v} \in V:\pijl{u}+\pijl{v}=\pijl{v}+\pijl{u}$
    \item $\forall r\in \mathbb{R}:\forall \pijl{u},\pijl{v} \in V:r\cdot (\pijl{u}+\pijl{v})=r\cdot \pijl{u}+r\cdot \pijl{v}$
    \item $\forall r,s\in \mathbb{R}:\forall \pijl{u} \in V:(r+s)\cdot \pijl{u}=r\cdot \pijl{u}+s \cdot \pijl{u}$
    \item $\forall r,s\in \mathbb{R}:\forall \pijl{u} \in V:r\cdot (s\cdot \pijl{u})=(r\cdot s) \cdot \pijl{u}$
    \item $\forall \pijl{u}\in V:1\cdot \pijl{u}=\pijl{u}$
\end{enumerate}
\end{definition}

\begin{exercise}
    \begin{question}
        Vormt de verzameling van alle magische vierkanten een vectorruimte?
    \end{question}
    \begin{question}
        Wat met de verzameling van alle magische vierkanten met magische constante \(3\) of constante \(0\)?
    \end{question}
    \begin{oplossing}
        Ja, de verzameling van alle magische vierkanten vormt een vectorruimte.

        De magische vierkanten met magische constante \(3\) vormen geen vectorruimte.
        Als twee vierkanten \(M\) en \(N\) elk een magische constante van \(3\) hebben, dan heeft hun som \(M + N\) een magische constante van \(6\), wat buiten de verzameling valt.

        De magische vierkanten met magische constante \(0\) vormen wel een vectorruimte.
    \end{oplossing}
\end{exercise}

\begin{exercise}
    Wat is het verband tussen
    \begin{itemize}
        \item de vectorruimte van magische vierkanten met magische constante \(0\): \(\mathbb{M}_{n\times n}^0\),
        \item de vectorruimte van alle magische vierkanten van \(3 \times 3\): \(\mathbb{M}^{3 \times 3}\) en
        \item de vectorruimte van alle \(3 \times 3\) matrices: \(\mathbb{R}^{3 \times 3}\) en
        \item de vectorruimte van alle \(n \times n\) matrices: \(\mathbb{R}^{n \times n}\)?
    \end{itemize}
    (allen voorzien van de gebruikelijke optelling en scalaire vermenigvuldiging)
    \begin{oplossing}
        We hebben de volgende inclusies:
        \begin{align*}
            \mathbb{M}_{n\times n}^0 &\subseteq \mathbb{R}^{n \times n}\\
            \mathbb{M}^{3 \times 3} &\subseteq \mathbb{R}^{3 \times 3}
        \end{align*}
        Andere verbanden zoals \(\mathbb{M}_{n\times n}^0 \subseteq \mathbb{M}^{3 \times 3}\) zijn er op zich niet onmiddellijk, aangezien de groottes van de magische vierkanten kunnen verschillen.
        Het vierkant \(
        \begin{array}{|c|c|}
        \hline
        0 & 0 \\
        \hline
        0 & 0 \\
        \hline
        \end{array}
        \)
        behoort bijvoorbeeld tot \(\mathbb{M}_{2\times 2}^0\), maar niet tot \(\mathbb{M}^{3 \times 3}\).
        Op dezelfde manier is bijvoorbeeld \(\mathbb{R}^2\) geen deelruimte van \(\mathbb{R}^{3}\), al kan je wel op intuitieve wijze zeggen dat \(\mathbb{R}^2\) inpasbaar is in \(\mathbb{R}^{3}\).
    \end{oplossing}
\end{exercise}

\begin{exercise}
    Wat kan je op basis van de vorige oefeningen zeggen over de dimensie van \(\mathbb{M}^{3 \times 3}\)?
    \begin{oplossing}
        De matrices \(A\), \(B\) en \(C\) uit de vorige oefeningen zijn lineair onafhankelijk.
        De dimensie van \(\mathbb{M}^{3 \times 3}\) is dus minstens \(3\).
        Omgekeerd kan je elk magisch vierkant van \(3 \times 3\) schrijven als een lineaire combinatie van \(A\), \(B\) en \(C\), al is dat niet zo evident (zie \href{https://math.stackexchange.com/questions/2137182/showing-a-set-of-matrices-is-a-basis-of-all-3x3-magic-squares}{deze link}).

        De dimensie van \(\mathbb{M}^{3 \times 3}\) is precies \(3\).
    \end{oplossing}
\end{exercise}

% optional: \(\mathbb{R}(\sqrt{2})\)! real space over numbers adjoined with root 2

\end{document}