\documentclass{ximera}
\input{../preamble}
\addPrintStyle{..}
\begin{document}
	\author{Wiskunde Op Maat}
	\xmtitle{Kwadratische vergelijkingen oplossen in de complexe getallen}{}


\begin{proposition}\nl

    Een (reële) tweedegraadsvergelijking van de vorm $ax^2+bx+c$, met $a,b,c\in\R$ en $a\neq0$
    heeft altijd twee complex toegevoegde oplossingen, namelijk 
    \begin{center}
        \important{x_1=\dfrac{-b+\sqrt{b^2-4ac}}{2a}} en \important{x_2=\dfrac{-b-\sqrt{b^2-4ac}}{2a}}
    \end{center}
    die echter samenvallen als $b^2-4ac=0$, en dan gelijk worden aan $x_1=x_2=\dfrac{-b}{2a}$.
\end{proposition}

\begin{exercise}
    
    
    \begin{question} De wortels van \( \quad  x^2 + 4x + 5 = 0    \) zijn  \(      \quad \answer[onlinenoinput]{ -2 + i                  \text{ en              }  -2 - i                  } \) \end{question}
    \begin{question} De wortels van \( \quad  x^2 - 6x + 10 = 0   \) zijn  \(      \quad \answer[onlinenoinput]{ 3 + i                   \text{ en              }  3 - i                   } \) \end{question}
    \begin{question} De wortels van \( \quad -2x^2 + 4x + 8  \)      zijn  \(      \quad \answer[onlinenoinput]{ -2                      \text{ en              } 4                        } \) \end{question}
    \begin{question} De wortels van \( \quad  x^2 + 2x + 2 = 0    \) zijn  \(      \quad \answer[onlinenoinput]{ -1 + i                  \text{ en              }  -1 - i                  } \) \end{question}
    \begin{question} De wortels van \( \quad 3x^2 - 6x + 3   \)      zijn  \(      \quad \answer[onlinenoinput]{ 1                       \text{ (dubbele wortel)}                          } \) \end{question}
    \begin{question} De wortels van \( \quad  3x^2 - 6x + 7 = 0   \) zijn  \(      \quad \answer[onlinenoinput]{ 1 + \frac{2i}{\sqrt{3}} \text{ en              }  1 - \frac{2i}{\sqrt{3}} } \) \end{question}
    \begin{question} De wortels van \( \quad  5x^2 + 10x + 15 = 0 \) zijn  \(      \quad \answer[onlinenoinput]{ -1 + i                  \text{ en              }  -1 - i                  } \) \end{question}
    \begin{question} De wortels van \( \quad  4x^2 + 4x + 5 = 0   \) zijn  \(      \quad \answer[onlinenoinput]{ -\frac{1}{2} + i        \text{ en              }  -\frac{1}{2} - i        } \) \end{question}
    \begin{question} De wortels van \( \quad  x^2 - 8x + 20 = 0   \) zijn  \(      \quad \answer[onlinenoinput]{ 4 + 2i                  \text{ en              }  4 - 2i                  } \) \end{question}
    \begin{question} De wortels van \( \quad x^2 - 4x + 3    \)      zijn  \(      \quad \answer[onlinenoinput]{ 1                       \text{ en              }3                         } \) \end{question}
    
    
\end{exercise}
    
    
    
    
\begin{exercise}

    
    \begin{question} De wortels van \( \quad  2x^2 + 4x + 5 = 0   \) zijn  \(      \quad \answer[onlinenoinput]{ -1 + i                             \text{ en }  -1 - i                             } \) \end{question}
    \begin{question} De wortels van \( \quad 2x^2 + 5x + 2   \)      zijn  \(      \quad \answer[onlinenoinput]{ -\frac{1}{2}                       \text{ en } -2                                  } \) \end{question}
    \begin{question} De wortels van \( \quad  2x^2 + 6x + 10 = 0  \) zijn  \(      \quad \answer[onlinenoinput]{ -1.5 + i                           \text{ en }  -1.5 - i                           } \) \end{question}
    \begin{question} De wortels van \( \quad  3x^2 + 9x + 15 = 0  \) zijn  \(      \quad \answer[onlinenoinput]{ -\frac{3}{2} + i\frac{\sqrt{3}}{2} \text{ en }  -\frac{3}{2} - i\frac{\sqrt{3}}{2} } \) \end{question}
    \begin{question} De wortels van \( \quad  x^2 + 6x + 10 = 0   \) zijn  \(      \quad \answer[onlinenoinput]{ -3 + i                             \text{ en }  -3 - i                             } \) \end{question}
    \begin{question} De wortels van \( \quad x^2 + 2x - 8    \)      zijn  \(      \quad \answer[onlinenoinput]{ -4                                 \text{ en } 2                                   } \) \end{question}
    \begin{question} De wortels van \( \quad  2x^2 - 4x + 6 = 0   \) zijn  \(      \quad \answer[onlinenoinput]{ 1 + i                              \text{ en }  1 - i                              } \) \end{question}
    \begin{question} De wortels van \( \quad  3x^2 + 12x + 20 = 0 \) zijn  \(      \quad \answer[onlinenoinput]{ -2 + \frac{2i}{\sqrt{3}}           \text{ en }  -2 - \frac{2i}{\sqrt{3}}           } \) \end{question}
    \begin{question} De wortels van \( \quad  4x^2 + 8x + 18 = 0  \) zijn  \(      \quad \answer[onlinenoinput]{ -1 + \frac{\sqrt{2}i}{2}           \text{ en }  -1 - \frac{\sqrt{2}i}{2}           } \) \end{question}
    \begin{question} De wortels van \( \quad  x^2 + 4x + 8 = 0    \) zijn  \(      \quad \answer[onlinenoinput]{ -2 + i                             \text{ en }  -2 - i                             } \) \end{question}
    
    
\end{exercise} 




\end{document}