\documentclass{ximera}
\input{../preamble}
\addPrintStyle{..}
\begin{document}
	\author{Wiskunde Op Maat}
	\xmtitle{Complexe getallen voorstelllen in het compexe vlak }{}


\begin{exercise} 

    \begin{question} \( z_1  = 3 + 4i                                                                                                                                                 \quad \answer[onlineshowanswerbutton]{ \text{Polaire vorm: } 5\left(\cos\left(\tan^{-1}\left(\frac{4}{3}\right)\right) + i\sin\left(\tan^{-1}\left(\frac{4}{3}\right)\right)\right)           } \) \end{question}
    \begin{question} \( z_2  = 5 \left( \cos\left(\frac{\pi}{4}\right) + i \sin\left(\frac{\pi}{4}\right) \right)                                                                     \quad \answer[onlineshowanswerbutton]{ \text{Cartesische vorm: } 3 + 4i                                                                                                                       } \) \end{question}
    \begin{question} \( z_3  = -2 + i                                                                                                                                                 \quad \answer[onlineshowanswerbutton]{ \text{Polaire vorm: } \sqrt{5}\left(\cos\left(\tan^{-1}\left(\frac{-1}{2}\right)\right) + i\sin\left(\tan^{-1}\left(\frac{-1}{2}\right)\right)\right)  } \) \end{question}
    \begin{question} \( z_4  = \sqrt{5} \left( \cos\left( \tan^{-1}\left(\frac{-1}{2}\right) \right) + i \sin\left( \tan^{-1}\left(\frac{-1}{2}\right) \right) \right)                \quad \answer[onlineshowanswerbutton]{ \text{Cartesische vorm: } -2 + i                                                                                                                       } \) \end{question}
    \begin{question} \( z_5  = 5                                                                                                                                                      \quad \answer[onlineshowanswerbutton]{ \text{Polaire vorm: } 5\left(\cos(0) + i\sin(0)\right)                                                                                                 } \) \end{question}
    \begin{question} \( z_6  = 5 \left( \cos(0) + i \sin(0) \right)                                                                                                                   \quad \answer[onlineshowanswerbutton]{ \text{Cartesische vorm: } 5                                                                                                                            } \) \end{question}
    \begin{question} \( z_7  = -3 - 4i                                                                                                                                                \quad \answer[onlineshowanswerbutton]{ \text{Polaire vorm: } 5\left(\cos(\pi + \tan^{-1}\left(\frac{4}{3}\right)) + i\sin(\pi + \tan^{-1}\left(\frac{4}{3}\right))\right)                     } \) \end{question}
    \begin{question} \( z_8  = 5 \left( \cos\left(\pi + \tan^{-1}\left(\frac{4}{3}\right)\right) + i \sin\left(\pi + \tan^{-1}\left(\frac{4}{3}\right)\right) \right)                 \quad \answer[onlineshowanswerbutton]{ \text{Cartesische vorm: } -3 - 4i                                                                                                                      } \) \end{question}
    \begin{question} \( z_9  = 2 - 2i                                                                                                                                                 \quad \answer[onlineshowanswerbutton]{ \text{Polaire vorm: } 2\sqrt{2}\left(\cos\left(\frac{-\pi}{4}\right) + i\sin\left(\frac{-\pi}{4}\right)\right)                                         } \) \end{question}
    \begin{question} \( z_10 = 2 \sqrt{2} \left( \cos\left( \frac{-\pi}{4} \right) + i \sin\left( \frac{-\pi}{4} \right) \right)                                                      \quad \answer[onlineshowanswerbutton]{ \text{Cartesische vorm: } 2 - 2i                                                                                                                       } \) \end{question}

\end{exercise}
\begin{exercise} 

    \begin{question} \( z_1  = 1 + i                                                                                                                                                  \quad \answer[onlineshowanswerbutton]{ \text{Polaire vorm: } \sqrt{2}\left(\cos\left(\frac{\pi}{4}\right) + i\sin\left(\frac{\pi}{4}\right)\right)                                            } \) \end{question}
    \begin{question} \( z_2  = \sqrt{2} \left( \cos\left(\frac{\pi}{4}\right) + i \sin\left(\frac{\pi}{4}\right) \right)                                                              \quad \answer[onlineshowanswerbutton]{ \text{Cartesische vorm: } 1 + i                                                                                                                        } \) \end{question}
    \begin{question} \( z_3  = -1                                                                                                                                                     \quad \answer[onlineshowanswerbutton]{ \text{Polaire vorm: } 1\left(\cos(\pi) + i\sin(\pi)\right)                                                                                             } \) \end{question}
    \begin{question} \( z_4  = 1 \left( \cos(\pi) + i \sin(\pi) \right)                                                                                                               \quad \answer[onlineshowanswerbutton]{ \text{Cartesische vorm: } -1                                                                                                                           } \) \end{question}
    \begin{question} \( z_5  = 0 + 5i                                                                                                                                                 \quad \answer[onlineshowanswerbutton]{ \text{Polaire vorm: } 5\left(\cos\left(\frac{\pi}{2}\right) + i\sin\left(\frac{\pi}{2}\right)\right)                                                   } \) \end{question}
    \begin{question} \( z_6  = 5 \left( \cos\left( \frac{\pi}{2} \right) + i \sin\left( \frac{\pi}{2} \right) \right)                                                                 \quad \answer[onlineshowanswerbutton]{ \text{Cartesische vorm: } 0 + 5i                                                                                                                       } \) \end{question}
    \begin{question} \( z_7  = -4 + 0i                                                                                                                                                \quad \answer[onlineshowanswerbutton]{ \text{Polaire vorm: } 4\left(\cos(\pi) + i\sin(\pi)\right)                                                                                             } \) \end{question}
    \begin{question} \( z_8  = 4 \left( \cos(\pi) + i \sin(\pi) \right)                                                                                                               \quad \answer[onlineshowanswerbutton]{ \text{Cartesische vorm: } -4 + 0i                                                                                                                      } \) \end{question}
    \begin{question} \( z_9  = 3i                                                                                                                                                     \quad \answer[onlineshowanswerbutton]{ \text{Polaire vorm: } 3\left(\cos\left(\frac{\pi}{2}\right) + i\sin\left(\frac{\pi}{2}\right)\right)                                                   } \) \end{question}
    \begin{question} \( z_10 = 3 \left( \cos\left( \frac{\pi}{2} \right) + i \sin\left( \frac{\pi}{2} \right) \right)                                                                 \quad \answer[onlineshowanswerbutton]{ \text{Cartesische vorm: } 3i                                                                                                                           } \) \end{question}

\end{exercise}
 \begin{exercise} 

    \begin{question} \( z_1  = 4 + 0i                                                                                                                                                 \quad \answer[onlineshowanswerbutton]{ \text{Polaire vorm: } 4\left(\cos(0) + i\sin(0)\right)                                                                                                 } \) \end{question}
    \begin{question} \( z_2  = 2 \left( \cos\left( \frac{\pi}{2} \right) + i \sin\left( \frac{\pi}{2} \right) \right)                                                                 \quad \answer[onlineshowanswerbutton]{ \text{Cartesische vorm: } 0 + 2i                                                                                                                       } \) \end{question}
    \begin{question} \( z_3  = 0 + 2i                                                                                                                                                 \quad \answer[onlineshowanswerbutton]{ \text{Polaire vorm: } 2\left(\cos\left(\frac{\pi}{2}\right) + i\sin\left(\frac{\pi}{2}\right)\right)                                                   } \) \end{question}
    \begin{question} \( z_4  = 4 \left( \cos(0) + i \sin(0) \right)                                                                                                                   \quad \answer[onlineshowanswerbutton]{ \text{Cartesische vorm: } 4 + 0i                                                                                                                       } \) \end{question}
    \begin{question} \( z_5  = -2 - 3i                                                                                                                                                \quad \answer[onlineshowanswerbutton]{ \text{Polaire vorm: } \sqrt{13}\left(\cos(\pi + \tan^{-1}\left(\frac{3}{2}\right)) + i\sin(\pi + \tan^{-1}\left(\frac{3}{2}\right))\right)             } \) \end{question}
    \begin{question} \( z_6  = \sqrt{13} \left( \cos\left( \pi + \tan^{-1}\left( \frac{3}{2} \right) \right) + i \sin\left( \pi + \tan^{-1}\left( \frac{3}{2} \right) \right) \right) \quad \answer[onlineshowanswerbutton]{ \text{Cartesische vorm: } -2 - 3i                                                                                                                      } \) \end{question}
    \begin{question} \( z_7  = 1 - 3i                                                                                                                                                 \quad \answer[onlineshowanswerbutton]{ \text{Polaire vorm: } \sqrt{10}\left(\cos\left(\tan^{-1}\left(\frac{-3}{1}\right)\right) + i\sin\left(\tan^{-1}\left(\frac{-3}{1}\right)\right)\right) } \) \end{question}
    \begin{question} \( z_8  = \sqrt{10} \left( \cos\left( \tan^{-1}\left( \frac{-3}{1} \right) \right) + i \sin\left( \tan^{-1}\left( \frac{-3}{1} \right) \right) \right)           \quad \answer[onlineshowanswerbutton]{ \text{Cartesische vorm: } 1 - 3i                                                                                                                       } \) \end{question}
    \begin{question} \( z_9  = 2 + 0i                                                                                                                                                 \quad \answer[onlineshowanswerbutton]{ \text{Polaire vorm: } 2\left(\cos(0) + i\sin(0)\right)                                                                                                 } \) \end{question}
    \begin{question} \( z_10 = 2 \left( \cos(0) + i \sin(0) \right)                                                                                                                   \quad \answer[onlineshowanswerbutton]{ \text{Cartesische vorm: } 2 + 0i                                                                                                                       } \) \end{question}   

\end{exercise}
 \begin{exercise} 

    \begin{question} \( z_1  = -1 + 0i                                                                                                                                                \quad \answer[onlineshowanswerbutton]{ \text{Polaire vorm: } 1\left(\cos(\pi) + i\sin(\pi)\right)                                                                                             } \) \end{question}
    \begin{question} \( z_2  = \sqrt{20} \left( \cos\left( \tan^{-1}\left( \frac{2}{-4} \right) \right) + i \sin\left( \tan^{-1}\left( \frac{2}{-4} \right) \right) \right)           \quad \answer[onlineshowanswerbutton]{ \text{Cartesische vorm: } -4 + 2i                                                                                                                      } \) \end{question}
    \begin{question} \( z_3  = -4 + 2i                                                                                                                                                \quad \answer[onlineshowanswerbutton]{ \text{Polaire vorm: } \sqrt{20}\left(\cos(\tan^{-1}\left(\frac{2}{-4}\right)) + i\sin(\tan^{-1}\left(\frac{2}{-4}\right))\right)                       } \) \end{question}
    \begin{question} \( z_4  = 1 \left( \cos(\pi) + i \sin(\pi) \right)                                                                                                               \quad \answer[onlineshowanswerbutton]{ \text{Cartesische vorm: } -1 + 0i                                                                                                                      } \) \end{question}
    \begin{question} \( z_5  = 0 + 4i                                                                                                                                                 \quad \answer[onlineshowanswerbutton]{ \text{Polaire vorm: } 4\left(\cos\left(\frac{\pi}{2}\right) + i\sin\left(\frac{\pi}{2}\right)\right)                                                   } \) \end{question}
    \begin{question} \( z_6  = 4 \left( \cos\left( \frac{\pi}{2} \right) + i \sin\left( \frac{\pi}{2} \right) \right)                                                                 \quad \answer[onlineshowanswerbutton]{ \text{Cartesische vorm: } 0 + 4i                                                                                                                       } \) \end{question}
    \begin{question} \( z_7  = 3 + 0i                                                                                                                                                 \quad \answer[onlineshowanswerbutton]{ \text{Polaire vorm: } 3\left(\cos(0) + i\sin(0)\right)                                                                                                 } \) \end{question}
    \begin{question} \( z_8  = 3 \left( \cos(0) + i \sin(0) \right)                                                                                                                   \quad \answer[onlineshowanswerbutton]{ \text{Cartesische vorm: } 3 + 0i                                                                                                                       } \) \end{question}
    \begin{question} \( z_9  = -5 + 5i                                                                                                                                                \quad \answer[onlineshowanswerbutton]{ \text{Polaire vorm: } 5\sqrt{2}\left(\cos\left(\frac{3\pi}{4}\right) + i\sin\left(\frac{3\pi}{4}\right)\right)                                         } \) \end{question}
    \begin{question} \( z_10 = 5 \sqrt{2} \left( \cos\left( \frac{3\pi}{4} \right) + i \sin\left( \frac{3\pi}{4} \right) \right)                                                      \quad \answer[onlineshowanswerbutton]{ \text{Cartesische vorm: } -5 + 5i                                                                                                                      } \) \end{question}

\end{exercise}

\end{document}