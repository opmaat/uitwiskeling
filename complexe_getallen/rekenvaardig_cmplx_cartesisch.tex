\documentclass{ximera}
\input{../preamble}
\addPrintStyle{..}
\begin{document}
	\author{Wiskunde Op Maat}
	\xmtitle{Complexe getallen voorstelllen in het complexe vlak}{}


\begin{exercise} Gegeven zijn volgende complexe getallen. Bereken. 
    \begin{itemize}
        \item \( z_1 = 3 + 4i  \)
        \item \( z_2 = -2 + i  \)
        \item \( z_3 = 3       \)
        \item \( z_4 = -1 - 3i \)
    \end{itemize}

\begin{question} \( z_1 +  z_4       =\answer[onlineshowanswerbutton]{} \) \end{question}    
\begin{question} \( z_2 \cdot  z_1   =\answer[onlineshowanswerbutton]{} \) \end{question} 
\begin{question} \( z_3 -  z_4       =\answer[onlineshowanswerbutton]{} \) \end{question}    
\begin{question} \( z_4  \cdot z_ 4  =\answer[onlineshowanswerbutton]{} \) \end{question}
\begin{question} \( \frac{z_2}{z_1}  =\answer[onlineshowanswerbutton]{} \) \end{question}
\begin{question} \( z_3  + z_2       =\answer[onlineshowanswerbutton]{} \) \end{question}
\begin{question} \( \frac{z_2}{z_2}  =\answer[onlineshowanswerbutton]{} \) \end{question}
\begin{question} \( z_2  \cdot z_3   =\answer[onlineshowanswerbutton]{} \) \end{question}
\begin{question} \( \frac{z_1}{z_ 3} =\answer[onlineshowanswerbutton]{} \) \end{question}
\begin{question} \( \frac{z_1}{z_2}  =\answer[onlineshowanswerbutton]{} \) \end{question}    


\end{exercise}

% HOE ZET IK DEZE UITWERKING ER BEST IN? 


% \begin{question} \( z_1 + z_4 = \)       \quad Answer: \( (3 + 4i) + (-1 - 3i) = 2 + i \) \end{question}    
% \begin{question} \( z_2 \cdot z_1 = \)   \quad Answer: \( (-2 + i) \cdot (3 + 4i) = -6 - 8i + 3i + 4i^2 = -6 - 5i - 4 = -10 - 5i \) \end{question} 
% \begin{question} \( z_3 - z_4 = \)       \quad Answer: \( 3 - (-1 - 3i) = 3 + 1 + 3i = 4 + 3i \) \end{question}    
% \begin{question} \( z_4 \cdot z_4 = \)   \quad Answer: \( (-1 - 3i) \cdot (-1 - 3i) = 1 + 6i + 9i^2 = 1 + 6i - 9 = -8 + 6i \) \end{question}
% \begin{question} \( \frac{z_2}{z_1} = \) \quad Answer: \( \frac{-2 + i}{3 + 4i} \). Multiply by the conjugate: \( \frac{(-2 + i)(3 - 4i)}{(3 + 4i)(3 - 4i)} = \frac{-6 + 8i + 3i - 4i^2}{9 + 16} = \frac{-6 + 11i + 4}{25} = \frac{-2 + 11i}{25} \) \end{question}
% \begin{question} \( z_3 + z_2 = \)       \quad Answer: \( 3 + (-2 + i) = 1 + i \) \end{question}
% \begin{question} \( \frac{z_2}{z_2} = \) \quad Answer: \( 1 \) \end{question}
% \begin{question} \( z_2 \cdot z_3 = \)   \quad Answer: \( (-2 + i) \cdot 3 = -6 + 3i \) \end{question}
% \begin{question} \( \frac{z_1}{z_3} = \) \quad Answer: \( \frac{3 + 4i}{3} = 1 + \frac{4i}{3} \) \end{question}
% \begin{question} \( \frac{z_1}{z_2} = \) \quad Answer: \( \frac{3 + 4i}{-2 + i} \). Multiply by the conjugate: \( \frac{(3 + 4i)(-2 - i)}{(-2 + i)(-2 - i)} = \frac{-6 - 3i - 8i - 4i^2}{4 + 1} = \frac{-6 - 11i + 4}{5} = \frac{-2 - 11i}{5} \) \end{question}






% \begin{exercise} Gegeven zijn volgende complexe getallen. Bereken. 
%     \begin{itemize}
%         \item \( z_ =  \)
%         \item \( z_ =  \)
%         \item \( z_ =  \)
%         \item \( z_ =  \)
%     \end{itemize}

% \begin{question} \( z_ +  z_       =\answer[onlineshowanswerbutton]{} \) \end{question}    
% \begin{question} \( z_ \cdot  z_   =\answer[onlineshowanswerbutton]{} \) \end{question} 
% \begin{question} \( z_ -  z_       =\answer[onlineshowanswerbutton]{} \) \end{question}    
% \begin{question} \( z_  \cdot z_4  =\answer[onlineshowanswerbutton]{} \) \end{question}
% \begin{question} \( \frac{z_}{z_}  =\answer[onlineshowanswerbutton]{} \) \end{question}
% \begin{question} \( z_  + z_       =\answer[onlineshowanswerbutton]{} \) \end{question}
% \begin{question} \( \frac{z_}{z_}  =\answer[onlineshowanswerbutton]{} \) \end{question}
% \begin{question} \( z_  \cdot z_   =\answer[onlineshowanswerbutton]{} \) \end{question}
% \begin{question} \( \frac{z_}{z_3} =\answer[onlineshowanswerbutton]{} \) \end{question}
% \begin{question} \( \frac{z_}{z_}  =\answer[onlineshowanswerbutton]{} \) \end{question}    


% \end{exercise}





\end{document}